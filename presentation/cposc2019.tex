\documentclass[]{beamer}
\usepackage[framemethod=tikz]{mdframed}
\usepackage{pgfplots}
\usepackage{tikz}
\usepackage{url}
\usepackage{xcolor}
\usepackage{xmpmulti}
\usetikzlibrary{shapes,arrows}
\hypersetup{pdfstartview={Fit}} % Fit the presentation to the window when first displayed
\usetheme{Frankfurt}
\pgfplotsset{compat=1.14}
\beamertemplatenavigationsymbolsempty{} % Remove Beamer navigation symbols
\nonstopmode{} % Keep making the file through errors
\usepackage[sfdefault]{sourcesanspro}
\usepackage[T1]{fontenc}

% http://danielfalster.com/blog/2013/06/18/a-nice-title-page-for-beamer-presentations/
\newmdenv[tikzsetting={draw=black,fill=white,fill opacity=0.7, line width=1pt},backgroundcolor=none,leftmargin=0,rightmargin=0,innertopmargin=5pt,skipbelow=\baselineskip,skipabove=\baselineskip]{TitleBox}

\title{Open Circuit: \\Printed Circuit Boards and Free Software}
\author[Swartz]{Tom~Swartz}
\institute{Central PA Open Source Conference 2019}
\date{September 21 2019}
\subject{Computer Science}
\begin{document}

% Title Page
{\usebackgroundtemplate{\includegraphics[height=1.00\paperwidth,keepaspectratio]{images/kicad3d.png}}
\begin{frame}[plain]
    \vspace{17em}
    \begin{TitleBox}
        \begin{center}
            {\color{red}\Large\inserttitle\color{black}}\\
        \end{center}
        \insertauthor{}\hfill\insertinstitute{}\\
        {\footnotesize
        \href{http://twitter.com/tswartz07}{@tswartz07}
        \hfill
        \href{mailto: tom@tswartz.net}{tom@tswartz.net}
        }
    \end{TitleBox}
\end{frame}}

\begin{frame}[plain]
    \frametitle{Who is this guy?}
    \framesubtitle{And what is he doing here?}
    \begin{columns}[T]
        \begin{column}[T]{5cm}
           {\huge Tom Swartz}
            \begin{itemize}%[<+->]
                \item{Hobbyist hardware hacker}
                \item{Make PCBs for various personal projects}
            \end{itemize}
        \end{column}
        \begin{column}[T]{5cm}
            \includegraphics[height=6cm]{images/me.jpg}
        \end{column}
    \end{columns}
\end{frame}
% Obligatory memes
\begin{frame}[plain]
    \includegraphics[width=0.85\paperwidth,keepaspectratio]{images/i_think_neat.png}
\end{frame}

% What's ahead?
\begin{frame}[plain]
    \frametitle{Agenda}
    \tableofcontents
\end{frame}

\section[Background]{Background}
\subsection[But Why?]{Why make your own circuit boards?}
\begin{frame}
    \frametitle{\insertsection{} Outline}
    \begin{itemize}
        \item For finishing projects
        \item Project evolution- Perfboard $\rightarrow{}$ PCB
        \item Intro to KiCAD
    \end{itemize}
\end{frame}

\begin{frame}
    \frametitle{But Why?}
    \framesubtitle{What does a PCB give my project?}
    \begin{itemize}[<+->]
        \item{Allows for (comparatively) smaller project footprints}
        \item{Allows for sensitive signals to maintain integrity}
        \item{Allows for easy duplication}
        \item{Allows far superior current carrying capacity}
        \item{Ensures product longevity}
        \item{Makes a project `finished'}
    \end{itemize}
\end{frame}

\begin{frame}
    \frametitle{Terminology}
    \begin{description}
        \item[PCB]{Printed Circuit Boards}
        \item[EDA]{Electronic Design Automation, a collection of tools to design
            and analyse semiconductors and printed circuit boards}
        \item[Gerber Files]{(Sometimes \emph{.gbr}) ASCII vector files indicating the method
            to construct a printed circuit board}
    \end{description}
\end{frame}

\subsection[KiCAD]{Introduction to KiCAD}
\begin{frame}
    \frametitle{\insertsection{} Outline}
    \begin{itemize}
        \item Intro to KiCAD
        \item Overview of EDA programs
    \end{itemize}
\end{frame}

\begin{frame}
    \frametitle{KiCAD}
    \includegraphics[width=0.85\paperwidth,keepaspectratio]{images/KiCad_Pcbnew_OpenGL.png}
\end{frame}

\begin{frame}
    \frametitle{KiCAD}
    \framesubtitle{About the program}
    \begin{itemize}
        \item{Free, Open Source EDA program}
        \item{27 years of continuous development}
        \item{Used by CERN}
        \item{Single suite to create PCBs from start to finish}
        \item{Nearly all project files can be stored in version control}
    \end{itemize}
\end{frame}

\begin{frame}
    \frametitle{KiCAD}
    \framesubtitle{About the program}
    The KiCAD suite has five main parts:
    \begin{description}
        \item[KiCad]{The project manager}
        \item[Eeschema]{The schematic capture editor}
        \item[Pcbnew]{The PCB layout program. It also has a 3D view}
        \item[GerbView]{The Gerber viewer}
        \item[Bmp2Comp]{Tool to convert images to footprints for PCB artwork}
    \end{description}
\end{frame}


\section[Design]{Designing a Schematic}
\begin{frame}
    \frametitle{\insertsection{} Outline}
    \begin{itemize}
        \item Start small
        \item Test prototype
        \item Get datasheets
    \end{itemize}
\end{frame}

\begin{frame}
    \frametitle{Designing a schematic}
    \begin{columns}[T]
        \begin{column}[T]{5cm}
            \begin{itemize}%[<+->]
                \item{Difficulty of this step varies, but is
                    typically the easiest}
                \item{Take existing circuit, or manufacturer
                    datasheets}
                \item{Place parts}
                \item{Hook up the dots}
            \end{itemize}
        \end{column}
        \begin{column}[T]{5cm}
            \includegraphics[height=7cm]{images/eeschema.png}
        \end{column}
    \end{columns}
\end{frame}

\section[Layout]{Creating a PCB Layout}
\begin{frame}
    \frametitle{\insertsection{} Outline}
    \begin{itemize}
        \item Lay out the rats nest
        \item Find the matching parts
        \item Place parts
        \item Zones- ground planes, power rails, etc
    \end{itemize}
\end{frame}

\section[Fabrication]{Preparing for Factory Fabrication}
\begin{frame}
    \frametitle{\insertsection{} Outline}
    \begin{itemize}
        \item Each PCB Fab house is different, but general rules apply
        \item Gerber Files
        \item General limits and tolerances
        \item Exporting files, loading them to Fab house
    \end{itemize}
\end{frame}

\section[Post-Production]{Post-Production}
\begin{frame}
    \frametitle{\insertsection{} Outline}
    \begin{itemize}
        \item General info on how quick process is
        \item Receiving the boards
        \item Testing boards for defects, issues from design
        \item Populating the boards
        \item Profit
    \end{itemize}
\end{frame}

\begin{frame}
    \frametitle{Post-Production}
    \framesubtitle{JLCPCB Manufacturing Process}
    \begin{center}
        \textbf{Order Placed to Out the Door:}\\19 Hours, 5 Minutes, 5 Seconds
        \includegraphics[height=0.70\paperheight,keepaspectratio]{images/fab_timeline.png}
    \end{center}
\end{frame}

\begin{frame}
    \frametitle{Post-Production}
    \framesubtitle{Receiving the PCBs}
    \begin{center}
        \includegraphics[height=0.70\paperheight,keepaspectratio]{images/packaging.jpg}
    \end{center}
\end{frame}
\begin{frame}
    \frametitle{Testing the boards}
    Important to test for missed design issues

    Use a multimeter to buzz it out
\end{frame}


\begin{frame}
    \frametitle{Post-Production}
    \framesubtitle{Receiving the PCBs}
    \begin{center}
        \includegraphics[height=0.70\paperheight,keepaspectratio]{images/complete.jpg}
    \end{center}
\end{frame}

\end{document}

% vim: set ts=4 sw=4:
